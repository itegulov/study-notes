\section{Ticket 1: Нетипизированное лямбда исчисление, теорема Чёрча-Россера}
\label{sec-2}
\subsection{Основные определения}
\label{sec-2-1}
\begin{eqnarray*}
Variable = \left\{v_0, v_1, v_2 \dots\right\}
\end{eqnarray*}
$Variable$ --- набор свободных переменных.
\begin{eqnarray*}
\Lambda^- &::=& Variable |
(\Lambda^- \Lambda^-) |
(\lambda Variable . \Lambda^-)
\end{eqnarray*}
$\Lambda^-$ --- пред-терм.
Конструкторы для пред-термов называются:
``Переменная'', ``Применение'' и ``Абстракция'' соответственно. \\
Ассоциативность пробела левая и тело абстракции имеет высший приоритет
(жадно набирает пока может).
\begin{example}
$\lambda x . a b c d e = \lambda x . ((((a b) c) d) e)$
\end{example}
Также допустимо писать $\lambda a b . A$ и это будет значить
$\lambda a . \lambda b . A$ (просто такой синтаксический сахар).

\begin{definition}
Определим функцию $freeVariables :: \Lambda^- \to [Variable]$. \\
Она будет находить все свободные переменные в пред-терме. \\
Более формально:
\begin{itemize}
\item $freeVariables(x) = \left\{x\right\}$
\item $freeVariables(A B) = freeVariables(A) \cup freeVariables(B)$
\item $freeVariables(\lambda x . A) = freeVariables(A) \setminus \left\{x\right\}$
\end{itemize}
Если $freeVariables(term) = \varnothing$, то терм $term$ называется замкнутым или
закрытым.
\end{definition}
\begin{definition}
Введём понятие ``подстановка''.
Будем обозначать $term[var:=otherTerm]$, если требуется в пред-терме $term$
подставить $otherTerm$ вместо всех свободных вхождений $var$. \\
Более формально:
\begin{itemize}
\item $x[x:=N] = N$
\item $y[x:=N] = y$
\item $(P Q)[x:=N] = P[x:=N] Q[x:=N]$
\item $(\lambda x . P)[x:=N] = \lambda x . P$
\item $(\lambda y . P)[x:=N] = \lambda z . P[y:=z][x:=N]$ если $y \in freeVariables(N)$
\item $(\lambda y . P)[x:=N] = \lambda y . P[x:=N]$ иначе
\end{itemize}
\end{definition}
\subsection{$\alpha$-эквивалентность и $\lambda$-термы}
\label{sec-2-2}
\begin{definition}
$\alpha$-эвивалентность ($=_\alpha$) -- наименьшее отношение на $\Lambda^-$,
что:
\begin{itemize}
\item $P =_\alpha P$ $\forall P$
\item $\lambda x . P =_\alpha \lambda y . P [x:=y]$ если $y \not \in freeVariables(P)$
\end{itemize}
Также введём следующую операцию: $[M]_\alpha = \left\{ N \in \Lambda^- | M =_\alpha N \right\}$
\end{definition}
\begin{definition}
$\Lambda = \Lambda^-_{=_\alpha} = \left\{ [M]_\alpha | M \in \Lambda^- \right\}$
\end{definition}
Отсюда и дальше под термами мы будем подразумевать лямбда-термы, а не 
пред-термы, если не написано иначе.
\begin{definition}
Свободные переменные у лямбда-термов определяются точно так же
как и у пред-термов
\end{definition}
\begin{definition}
Подстановка в лямбда-термы определяются точно так же за исключением
подстановки в применение: \\
$(\lambda y . P)[x:=N] = \lambda y . P[x:= N]$, если $x \not= y$,
где $y \not \in freeVariables(N)$
\end{definition}
\subsection{$\beta$-редукция и её свойства}
\label{sec-2-3}
\begin{definition}
$\to_\beta$ --- наименьшее отношение на $\Lambda$, что:
\begin{itemize}
\item $\lambda x . P \to_\beta \lambda x . P'$, если $P \to_\beta P'$
\item $(P Z) \to_\beta (R W)$, если $P \to_\beta R$ и $Z = W$ или 
$Z \to_\beta W$ и $P = R$
\item $(\lambda x . P) Q \to_\beta P[x:=Q]$, если $Q$ свободна для
подстановки в $P$ относительно $x$
\end{itemize}
При этом терм вида $(\lambda x . P) Q$ называется $\beta$-редуксом,
а $P[x:=Q]$ --- $\beta$-\myworries{контрактумом}.
\end{definition}
\begin{definition}
Терм $T$ находится в нормальной форме, если $\not \exists T'$, что $T \to_\beta T'$.
\end{definition}
\begin{definition}
Мультишаговая $\beta$-редукция: \\
$A \twoheadrightarrow_\beta C$, если $\exists B_0 \dots B_n$, что
$A \to_\beta B_0 \to_\beta \dots \to_\beta B_n \to_\beta C$.
\end{definition}
Некоторые свойства $\twoheadrightarrow_\beta$:
\begin{itemize}
\item Рефлексивность: $A \twoheadrightarrow_\beta A$
\item Транзитивность
\item $A \twoheadrightarrow_\beta A' \Rightarrow
(A B) \twoheadrightarrow_\beta (A' B)$,
$(B A) \twoheadrightarrow_\beta (B A')$,
$\lambda x . A \twoheadrightarrow_\beta \lambda x . A'$
\item $A \twoheadrightarrow_\beta A' \Rightarrow
A[x:=B] \twoheadrightarrow_\beta A'[x:=B]$
\item $A \twoheadrightarrow_\beta A' \Rightarrow
C[x:=A] \twoheadrightarrow_\beta C[x:=A']$
\end{itemize}
\begin{proof}
Из всех пунктов неочевиден только 4-ый.
Без потери общности докажем, что $A \to_\beta A' \Rightarrow
A[x:=B] \twoheadrightarrow_\beta A'[x:=B]$.
Будем также считать, что все связанные переменные отличны от $x$.
Используем индукцию по разности длин терма и редукса. \\
База: Разность $= 0$
\begin{align}
& A = (\lambda y . P) Q \\
& A' = P[y:=Q] \\
& \text{Тогда } A[x:=B] = (\lambda y . P[x:=B]) Q[x:=B] \twoheadrightarrow_\beta \\
& (P[x:=B])[y:=Q[x:=B]] = P[y:=Q][x:=B] = A'[x:=B] \\
& \text{То есть } A[x:=B] \twoheadrightarrow_\beta A'[x:=B].
\end{align}
Переход: Раность $> 0$ \\
Рассмотрим три случая:
\begin{enumerate}
\item $A = A_0 A_1$ \\
Редекс находится внутри $A_0$ и по предположению индукции $A_0[x:=B]
\twoheadrightarrow_\beta A_0'[x:=B]$, тогда очевидным образом основное
утверждение верно.
\item $A = A_0 A_1$ \\
Редекс находится внутри $A_1$, а остальное аналогично пункту 1.
\item $A = \lambda z . A_0$ \\
Тогда из предположения индукции $A_0[x:=B] \to A_0'[x:=B]$ и
очевидно утверждение верно (используя правила подстановки).
\end{enumerate}
\end{proof}
\subsection{Теорема Чёрча-Россера}
\label{sec-2-4}
\begin{theorem}
$\forall M_1, M_2, M_3 \in \Lambda$:
если $M_1 \twoheadrightarrow_\beta M_2$ и $M_1 \twoheadrightarrow_\beta M_3$,
то существует $M_4$, что $M_2 \twoheadrightarrow_\beta M_4$ и $M_3
\twoheadrightarrow M_4$.
\end{theorem}
Для доказательства теоремы введём новое отношение и сформулируем несколько
вспомогательных лемм.
\begin{definition}
Отношение $\rightrightarrows$ (параллельная редукция) на $\Lambda$:
\begin{itemize}
\item $P \rightrightarrows P$
\item $P \rightrightarrows P' \Rightarrow \lambda x . P \rightrightarrows
\lambda x . P'$
\item $P \rightrightarrows P' \& Q \rightrightarrows Q' \Rightarrow
P Q \rightrightarrows P' Q'$
\item $P \rightrightarrows P' \& Q \rightrightarrows Q' \Rightarrow
(\lambda x . P) Q \rightrightarrows P'[x:=Q']$
\end{itemize}
Замечание: транзитивность отсутствует
\end{definition}
