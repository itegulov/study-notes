\section{Ticket 1: Нетипизированное лямбда исчисление, теорема Чёрча-Россера}
\label{sec-2}
\subsection{Основные определения}
\label{sec-2-1}
\begin{eqnarray*}
$Variable = {v_0, v_1, v_2 \dots}$
\end{eqnarray*}
$Variable$ --- набор свободных переменных.
\begin{eqnarray*}
\Lambda^- &::=& Variable |
(\Lambda^- \Lambda^-) |
(\lambda Variable . \Lambda^-)
\end{eqnarray*}
$\Lambda^-$ --- пред-терм.
Конструкторы для пред-термов называются:
``Переменная'', ``Применение'' и ``Абстракция'' соответственно. \\
Ассоциативность пробела левая и тело абстракции имеет высший приоритет
(жадно набирает пока может).
\begin{example}
$\lambda x . a b c d e = \lambda x . ((((a b) c) d) e)$
\end{example}
Также допустимо писать $\lambda a b . A$ и это будет значить
$\lambda a . \lambda b . A$ (просто такой синтаксический сахар).

\begin{definition}
Определим функцию $freeVariables :: \Lambda^- \to [Variable]$. \\
Она будет находить все свободные переменные в пред-терме. \\
Более формально:
\begin{itemize}
\item $freeVariables(x) = \left\{x\right\}$
\item $freeVariables(A B) = freeVariables(A) \cup freeVariables(B)$
\item $freeVariables(\lambda x . A) = freeVariables(A) \setminus \left\{x\right\}$
\end{itemize}
Если $freeVariables(term) = \varnothing$, то терм $term$ называется замкнутым или
закрытым.
\end{definition}
\begin{definition}
Введём понятие ``подстановка''.
Будем обозначать $term[var:=otherTerm]$, если требуется в пред-терме $term$
подставить $otherTerm$ вместо всех свободных вхождений $var$. \\
Более формально:
\begin{itemize}
\item $x[x:=N] = N$
\item $y[x:=N] = y$
\item $(P Q)[x:=N] = P[x:=N] Q[x:=N]$
\item $(\lambda x . P)[x:=N] = \lambda x . P$
\item $(\lambda y . P)[x:=N] = \lambda z . P[y:=z][x:=N]$ если $y \in freeVariables(N)$
\item $(\lambda y . P)[x:=N] = \lambda y . P[x:=N]$ иначе
\end{itemize}
\end{definition}