\section{Предисловие и источники}
\label{sec-1}
\subsection{Предисловие}
\label{sec-1-1}
Я не отвечаю за верность написанного - много информации я придумал сам, много
достал из недостоверных источников.
\subsection{Источники}
\begin{enumerate}
  \item Лекции, данные напрямую Штукенбергом Д.Г.
  \item \href{https://github.com/shd/tt2014/blob/master/conspect.pdf}{Конспект Штукенберга Д.Г.}
  \item \href{http://disi.unitn.it/~bernardi/RSISE11/Papers/curry-howard.pdf}{Sørensen, Urzyczyn. Lectures on the Curry-Howard Isomorphism}
  \item \href{http://www.compsciclub.ru/csclub/sites/default/files/slides/20110313_systems_of_typed_lambda_calculi_moskvin_lecture06.pdf}{ПОМИ РАН презентация}
  \item \href{http://www.nsl.com/misc/papers/martelli-montanari.pdf}{Efficient unification algorithm}
  \item \href{http://starling.rinet.ru/~goga/tapl/tapl.pdf}{Пирс. Типы в языках программирования.}
  \item \href{http://math.nsc.ru/~asm256/lambda/LambdaDec2012.pdf}{Денотационная семантика Лямбда-исчисления}
  \item \href{http://r.duckduckgo.com/l/?kh=-1\&uddg=http://homepages.inf.ed.ac.uk/wadler/papers/lineartaste/lineartaste-revised.pdf}{The taste of linear logic}
  \item \href{http://plato.stanford.edu/entries/logic-linear/}{Stanford encyclopedia -- linear logic}
  \item \href{http://webdoc.sub.gwdg.de/ebook/serien/ah/UU-CS/2002-031.pdf}{Generalizing Hindley-Milner Type Inference Algorithms}
  \item \href{http://gallium.inria.fr/~fpottier/publis/emlti-final.pdf}{The essence of ML type inference (for H-M constraints)}
\end{enumerate}